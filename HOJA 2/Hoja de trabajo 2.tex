\documentclass{article}

\usepackage{fancyhdr} % Required for custom headers
\usepackage{lastpage} % Required to determine the last page for the footer
\usepackage{extramarks} % Required for headers and footers
\usepackage[usenames,dvipsnames]{color} % Required for custom colors
\usepackage{graphicx} % Required to insert images
\usepackage{listings} % Required for insertion of code
\usepackage{courier} % Required for the courier font
\usepackage{multirow}
\usepackage{hyperref}
\usepackage{amsmath}
\usepackage{amssymb}

% Margins
\topmargin=-0.45in
\evensidemargin=0in
\oddsidemargin=0in
\textwidth=6.5in
\textheight=9.0in
\headsep=0.25in

\linespread{1.1} % Line spacing

%----------------------------------------------------------------------------------------
%	CODE INCLUSION CONFIGURATION
%----------------------------------------------------------------------------------------

\definecolor{MyDarkGreen}{rgb}{0.0,0.4,0.0} % This is the color used for comments
\lstloadlanguages{c} % Load Perl syntax for listings, for a list of other languages supported see: ftp://ftp.tex.ac.uk/tex-archive/macros/latex/contrib/listings/listings.pdf
\lstset{language=[sharp]c, % Use Perl in this example
        frame=single, % Single frame around code
        basicstyle=\small\ttfamily, % Use small true type font
        keywordstyle=[1]\color{Blue}\bf, % Perl functions bold and blue
        keywordstyle=[2]\color{Purple}, % Perl function arguments purple
        keywordstyle=[3]\color{Blue}\underbar, % Custom functions underlined and blue
        identifierstyle=, % Nothing special about identifiers                                         
        commentstyle=\usefont{T1}{pcr}{m}{sl}\color{MyDarkGreen}\small, % Comments small dark green courier font
        stringstyle=\color{Purple}, % Strings are purple
        showstringspaces=false, % Don't put marks in string spaces
        tabsize=5, % 5 spaces per tab
        %
        % Put standard Perl functions not included in the default language here
        morekeywords={rand},
        %
        % Put Perl function parameters here
        morekeywords=[2]{on, off, interp},
        %
        % Put user defined functions here
        morekeywords=[3]{test},
       	%
        morecomment=[l][\color{Blue}]{...}, % Line continuation (...) like blue comment
        numbers=left, % Line numbers on left
        firstnumber=1, % Line numbers start with line 1
        numberstyle=\tiny\color{Blue}, % Line numbers are blue and small
        stepnumber=5 % Line numbers go in steps of 5
}

\newcommand{\horrule}[1]{\rule{\linewidth}{#1}}

% Creates a new command to include a perl script, the first parameter is the filename of the script (without .pl), the second parameter is the caption
\newcommand{\perlscript}[2]{
\begin{itemize}
\item[]\lstinputlisting[caption=#2,label=#1]{#1.cs}
\end{itemize}
}

\begin{document}

\begin{tabular}{l l}
\multirow{1}{*}
 & Universidad del Istmo de Guatemala \\
 & Facultad de Ingenieria \\
 & Ing. en Sistemas \\
 & Informatica 1 \\
 & Jorge Luis Ortiz 
\end{tabular}
\\

\begin{center}
        \horrule{0.5pt}
        \huge{Tarea 2} \\
        
        \horrule{1pt}
\end{center}

\emph{Instrucciones: Resolver cada uno de los ejercicios siguiendo sus respectivas
instrucciones. El trabajo debe ser entregado a traves de Github, en su repositorio del curso, colocado en una carpeta llamada "Hoja de trabajo 1".
Al menos que la pregunta indique diferente, todas las respuestas a preguntas escritas deben presentarse en
un documento formato pdf, el cual haya sido generado mediante Latex. }

% \perlscript{homework_example}{Sample Perl Script With Highlighting}
\section*{Ejercicio \#1}
Demostrar utilizando inducci\'on:
\[
        \forall\ n.\ n^3\geq n^2
\]
\\donde $n\in\mathbb{N}$
\\
\ Primero aplicamos el caso base, n=0
\[ (0)^3 \geq (0)^2\]
\[ 0 \geq 0 \]
\ se cumple con el caso base.
\\{\bf Por lo tanto:} esta desigualdad se cumple para n y se demostrara para su sucesor, osea n+1
\[
  (n+1)^3\geq (n+1)^2 
\]
\ Esto es lo mismo que decir:
\[
(n+1)(n+1)^2\geq (n+1)^2
\]
\ Se pasa a dividir $(n+1)^2$
\[
(n+1) \geq (n+1)^2 / (n+1)^2
\]
\ La division da 1
\[
n+1\geq 1
\]
\ Se pasa a restar el 1 al lado derecho
\[
n \geq 1-1
\]
\ Esto da como resultado:
\[
n \geq 0
\]
\ Entonces n sera mayor o igual a 0, por consiguiente se cumple que  $n\in\mathbb{N}$
\section*{Ejercicio \#2}
Demostrar utilizando inducci\'on la desigualdad de Bernoulli:
\[
        \forall\ n.\ (1+x)^n\geq nx
\]
\\donde $n\in \mathbb{N}$, $x\in \mathbb{Q}$ y $x\geq -1$
\\
donde el lado izquierdo es mayor que $nx + 1$. 
\\
\\Podemos decir que si $x \geq -1$ entonces $x+1 \geq 0$ 
\\
\begin{itemize}
\item Aplicamos el caso base donde n=0
\[ (1+x)^0 \geq 0x\]
\[ 1 \geq 0 \]
\ donde 1 es mayor o igual a 0.
\\ 
\item Demostracion por hipotesis inductiva (n+1), para llegar a esto, se opera de la siguiente manera: 
\[ (1+x)^n * (1+x) \geq (nx+1)* (1+x) \]
\[ (1+x)^{n+1} \geq nx+x+1+nx^2 \]
\ se sabe que $nx^2 \geq 0$ , y por transitividad se tiene:
\[ (1+x)^{n+1} \geq nx+x+1+nx^2 \geq nx+x+1\] 
 \[ (1+x)^{n+1} \geq nx+x+1\]
\[ (1+x)^{n+1} \geq x(n+1)+1 \]
\ Por lo tanto, la formula es cierta para (n+1)



\end{itemize}

% \bibliography{../../recursos/referencias}
% \bibliographystyle{plain}

\end{document}